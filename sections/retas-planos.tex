\section{Retas}

\item Determine se são verdadeiras ou falsas as seguintes afirmações.
\begin{enumerate}[leftmargin=*]
\item  Duas retas paralelas a uma terceira são paralelas.
\item  Duas retas perpendiculares a uma terceira são paralelas.
\item  Duas retas paralelas a um plano são paralelas.
\item  Duas retas perpendiculares a um plano são paralelas.
\item  Duas retas ou se interceptam ou são paralelas.
\end{enumerate}

\item[\textcolor{blue}{44-45}] Obtenha equações paramétricas para a reta que passa pelos pontos
$P$ e $Q$ e também para o segmento de reta ligando esses pontos.
\item 
\begin{enumerate}[leftmargin=*]
    \item $P(3, -2),\quad Q(5, 1)$
    \item $P(5, -2, 1),\quad Q(2, 4, 2)$
\end{enumerate}
\item 
\begin{enumerate}[leftmargin=*]
    \item $P (-1, 3, 5), \quad Q(-1, 3, 2)$
    \item $P (-1, 3, 5),\quad Q(-1, 3, 2)$
\end{enumerate}
\item[\textcolor{blue}{46-50}] Obtenha equações paramétricas da reta que satisfaz as condições dadas.

\item A reta que passa por $(-5, 2)$ e é paralela a $\vb=2\ib - 3\jb$.
\item A reta que é tangente ao círculo $x^2 + y^2 = 25$ no ponto $(3,-4)$.
\item A reta que é tangente à parábola $y = x^2$ no ponto $(-2, 4)$.

\item A reta que passa por $(-2, 0, 5)$ e é paralela à reta $x = 1 + 2t$, $y = 4 - t$, $z = 6 + 2t$.
\item A reta que passa pela origem e é paralela à reta $x = t$, $y = -1+ t$, $z = 2$.

\item Em que ponto a reta $x = 1 + 3t$, $y = 2 - t$ intersecta
\begin{enumerate}[leftmargin=*]
\begin{multicols}{3}
    \item o eixo $x$
    \item o eixo $y$
    \item a parábola $y=x^2$?
    \end{multicols}
\end{enumerate}
\item Encontre a intersecção da reta $x = -2$, $y = 4 + 2t$, $z = -3 + t$ com os planos $xy$ e $xz$.

\item Determine a projeção ortogonal do ponto $P(2,-1, 3)$ sobre a reta $x=3t,\, y=-7+5t,\, z=2+2t$.


\item Considere o triângulo de vértices $A(1,0,-2)$, $B(2,-1,-6)$ e $(-4, 5, 2)$. Determine as equações paramétricas da reta suporte da mediana relativa ao lado $BC$.

\item Considere o triângulo de vértices  $A(3,3,3)$ , $B(0, 1, 3)$ e $C(6, 15, -3)$ . Determine as equações paramétricas da reta suporte da altura relativa ao lado $BC$.

\item[\textcolor{blue}{56-60}] Determine a posição relativa das retas $r_1$ e $r_2$.

\item $r_1: x = 2 + t,\, y = 2 + 3t,\, z = 3 + t$ \\ $r_2: x = 2 + s, \,y = 3 + 4s,\, z = 4 + 2s$

\item $r_1 : x = 1 + 7t, \,y = 3 + t, \,z = 5 - 3t$\\
$r_2 : x = 4 - s, \,y = 6, \,z = 7 + 2s$    

\item $r_1 : x = 2 -t, \,y = 3 + 2t, \,z = 1 + t$\\
$r_2 : x = 5 - 2s, \,y = 2+4s, \,z = 1 + 2s$ 

\item $r_1 : x = 2 +t, \,y = -3 -t, \,z = t$\\
$r_2 : x = \frac{1}{2}(3s + 1), \,y = s-1, \,z = \frac{1}{3}s$   
\item $r_1 : x = 2 +t, \,y = 4 - 2t, \,z = 1 + 3t$\\
$r_2 : x = -1 + 4s, \,y = 3-s, \,z = 2 + 2s$ 

\item Determine a distância do ponto $A(-2,1,2)$ à reta determinada pelos pontos $P(1,2,1)$ e $Q(0,-1,3)$.
\item Determine a medida da projeção ortogonal de $\vb=\ib + 2\jb + \kb$ sobre a reta $x=1-2t$,$y=t$, $z=-1-2t$.
\section{Planos}
\item Determine se são verdadeiras ou falsas as seguintes afirmações.
\begin{enumerate}[leftmargin=*]
    \item  Dois planos paralelos a um terceiro são paralelos.
    \item  Dois planos perpendiculares a um terceiro são paralelos.
    \item  Dois planos paralelos a uma reta são paralelos.
    \item  Dois planos perpendiculares a uma reta são paralelos.
    \item  Dois planos ou se interceptam ou são paralelos.
    \item  Um plano e uma reta ou se interceptam ou são paralelos.
\end{enumerate}


\item Determine uma equação do plano nos casos citados.

\begin{enumerate}[leftmargin=*]
\item passa pelo ponto $P(2, 6, 1)$ e tem $\mathbf{n}= \lan 1, 4, 2\ran$ como um vetor normal. 

\item passa pelo ponto $P(-1, -1, 2)$ e tem $\mathbf{n}=\lan -1, 7, 6\ran$ como um vetor normal. 

\item que passa pelos pontos $A(-2, 1, 1)$, $B(0, 2, 3)$ e $C(1, 0, -1)$.

\end{enumerate}
\item Determine uma equação do plano nos seguintes casos:
\begin{enumerate}[leftmargin=*]
    \item paralelo ao plano $2x-3y-z+5=0$ e que passa pelo ponto $P(4,-1,2)$.
    \item perpendicular à reta $x=1-3t$, $y=5+2t$, $z=-t$ e que passa pelo ponto $P(4,-1,2)$.
    \item determinado pelas retas  $x=1+2t$, $y=4t$, $z=-1+6t$ e  $x=s$, $y=1+2s$, $z=-2+3s$
    \item perpendicular ao eixo $y$ e que passa pelo ponto $P(-1,0,2)$.
    \item determinado pelo ponto $P(3,-1,2)$ e pela reta $x=t$,$y=2-t$, $z=3+2t$.
\end{enumerate}


\item Determine uma equação do plano determinado pelo ponto $P(3,-2,-1)$ e pela reta de intersecção dos planos
            $x+2y+z-1=0$ e   $2x+y-z+7=0$.
\item  Determine uma equação do plano determinado pelo ponto $P(1,2,1)$ e pela reta de intersecção dos planos $x-2y+z-3=0$ e $x=0$.

\item Determine uma equação do plano que contém o ponto $P(2, 0, 3)$ e a reta $x = -1 + t$, $y = t$, $z = -4 + 2t$.

\item[\textcolor{blue}{69-70}] Obtenha equações paramétricas da reta de interseção dos planos dados.

\item $\pi_1: -2x + 3y + 7z + 2 = 0$\\
$\pi_2: x + 2y - 3z + 5 = 0$
\item $\pi_1: 3x - 5y + 2z = 0$\\
$\pi_2: z=0$
\item Determine a distância entre o ponto $P(1, -2, 3)$ e o plano $2x - 2y + z = 4$.
\item Determine a distância entre os planos paralelos $-2x + y + z = 0$ e $6x - 3y - 3z - 5 = 0$.

\section{Problemas Suplementares II}
\item Mostre que a distância entre os planos paralelos 
\begin{align*}
    \pi_1: ax+by+cz + d_1=0\\
    \pi_2: ax+by+cz + d_2=0
\end{align*}
é 
\begin{align*}
    D=\dfrac{|d_1-d_2|}{\sqrt{a^2+b^2+ c^2}}
\end{align*}
\item Se $a$, $b$ e $c$ não são todos nulos, mostre que a equação $ax + by + cz+ d = 0$ representa um plano e $\lan a, b, c\ran$ é o vetor normal ao plano.

\item Determine uma equação do plano cujos pontos são equidistantes de $P_1(2, -1, 1)$ e $P_2(3, 1, 5)$.

\item Determine a equação paramétrica da $r$ que passa pelo ponto $P(1,2,0)$ e é paralela à reta de intersecção dos planos $2x-y-z+1=0$ e $x+3y+2z-4=0$
\item Suponha que $\vb_1$ e $\vb_2$ sejam vetores com $\|\vb_1\| = 2$, $\| \vb_2\| = 3$ e $\vb_1 \cdot \vb_2 = 5$. Seja $\vb_3 = \mathrm{proj}_{\vb_1} \vb_2$ , $\vb_4 = \mathrm{proj}_{\vb_2} \vb_3$, $\vb_5 = \mathrm{proj}_{\vb_3} \vb_4$  e assim por diante. Calcule $\sum_{n=1}^{\infty}\|\vb_n \|$.
\item Determine uma equação da esfera com centro $(2, 1, -3)$ que é tangente ao plano $x - 3y + 2z = 4$.

\item Determine a distância entre as retas reversas 
\begin{align*}
    r_1: & x = 1 + 7t, \, y = 3 + t, \, z = 5 - 3t\\
    r_2: & x = 4 - s,\, y = 6,\, z = 7 + 2s
\end{align*}
$$.$$